\documentclass[a4paper]{article}
\usepackage{amsmath}
\usepackage{amssymb}
\usepackage{commath}
\usepackage{graphicx}


\newcommand{\cm}[1]{\mathcal{#1}}


\title{Shielding and Orientation}
\date{2013 August 29}



\begin{document}
\maketitle

\section{Shielding}
\subsection{Concern}

Noise from muons stopping in things other than target will produce
avoidable background, such as high rates in detectors,
especially of protons from materials other than that of the target.
To the best of my ability, I have reproduced the
important aspects of the old detector in figure \ref{fig:olddet}.

\begin{figure}
  \includegraphic[scale=0.1]{img/olddet.jpg}
  \caption{The old detector.}
  \label{fig:olddet}
\end{figure}

The biggest issue would be the position of the collimator.
There there is a line of sight between scattered muons from
the beam window and the downstream detectors. We see the muon
stopping locations in \ref{fig:olddet_mustops}. This is
unnacceptable, so we put in the shielding we discussed,
as seen in \ref{fig:newdet}. Here we see the muons stopping
in the downstream arm are quite reduced, and all of the ones left
have scattered from the target (which we can do nothing about).
Specifically \ref{tab:percentstop}.

\begin{table}
  \begin{tabular}{l | c c}
    & Old & New \\
    \hline
    Downstream Detector & ??\% & ??\% \\
    Chamber & ??\% & ??\%
  \end{tabular}
\caption{Muon stop locations}
\label{tab:percentstop}
\end{table}

But what about the muons that are still getting through to the chamber? Well \ref{fig:walldet}. We see
the proton rates compared to before in \ref{tab:protrates}

\begin{table}
\begin{tabular}{l | c c }
  & New & New (with Lead Wall) \\
  \hline
  
\end{table}

\end{document}
